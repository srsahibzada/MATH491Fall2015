\documentclass[11pt]{article}
\usepackage{lineno}
\usepackage{graphicx}
\usepackage{bm}
\usepackage{graphicx}
\usepackage{amsmath}


\begin{document}





\DeclareGraphicsExtensions{.png,.jpg}



\begin{titlepage}

\newcommand{\HRule}{\rule{\linewidth}{0.5mm}} 

\center % Center everything on the page
 
%----------------------------------------------------------------------------------------
%	HEADING SECTIONS
%----------------------------------------------------------------------------------------

\textsc{\LARGE Texas A$\&$M University}\\[1.5cm] 
\textsc{\Large Research Project 1}\\[0.5cm] % 

%----------------------------------------------------------------------------------------
%	TITLE SECTION
%----------------------------------------------------------------------------------------

\HRule \\[0.4cm]
{ \huge \bfseries Some Conjectures on Fibonacci Numbers }\\[0.4cm]  %how do you make subtitles this is way too long
\HRule \\[1.5cm]
 
%----------------------------------------------------------------------------------------
%	AUTHOR SECTION
%----------------------------------------------------------------------------------------

\begin{minipage}{0.4\textwidth}
\begin{flushleft} \large
\emph{Authors:}\\
Daniel \textsc{Whatley}\\
Sarah \textsc{Sahibzada}\\
Taylor \textsc{Wilson}
\end{flushleft}
\end{minipage}
~
\begin{minipage}{0.4\textwidth}
\begin{flushright} \large
\emph{Supervisor:} \\
Dr. Sara \textsc{Pollock} 
\end{flushright}
\end{minipage}\\[4cm]

%----------------------------------------------------------------------------------------
%	DATE SECTION
%----------------------------------------------------------------------------------------

{\large \today}\\[3cm] 

%----------------------------------------------------------------------------------------
%	LOGO SECTION
%----------------------------------------------------------------------------------------

%\includegraphics[scale=.3]{tamulogo.png}\\[1cm] 
 
%----------------------------------------------------------------------------------------

\vfill 

\end{titlepage}

\tableofcontents
\newpage
\newpage

\section{Introduction}

The \textit{Fibonacci sequence}, $\{F_n\}$, is defined as follows: \[ F_1 = 1, \qquad F_2 = 1, \qquad F_n = F_{n - 1} + F_{n - 2} \quad \text{for} \quad n \geq 3. \] In this paper we analyze two conjectures: one about which Fibonacci numbers are perfect squares, and one about factorials (put reference and more info here).

\section{Theoretical Analysis}

M. Wunderlich (reference here) described an ingenious ``exclusion'' method to calculate which Fibonacci numbers, out of the first million, can possibly be perfect squares. The method is to take a series of primes, then 

We also created a data structure to check factorial conjecture... (insert more detail)

\section{Computational Approach}

\section{Results}

\section{Conclusion}

\section{Individual Contributions}

\section{References}


\end{document}
