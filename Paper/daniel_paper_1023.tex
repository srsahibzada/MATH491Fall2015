\documentclass[11pt]{article}
\usepackage{lineno}
\usepackage{graphicx}
\usepackage{bm}
\usepackage{graphicx}
\usepackage{amsmath}


\begin{document}





\DeclareGraphicsExtensions{.png,.jpg}



\begin{titlepage}

\newcommand{\HRule}{\rule{\linewidth}{0.5mm}} 

\center % Center everything on the page
 
%----------------------------------------------------------------------------------------
%	HEADING SECTIONS
%----------------------------------------------------------------------------------------

\textsc{\LARGE Texas A$\&$M University}\\[1.5cm] 
\textsc{\Large Research Project 1}\\[0.5cm] % 

%----------------------------------------------------------------------------------------
%	TITLE SECTION
%----------------------------------------------------------------------------------------

\HRule \\[0.4cm]
{ \huge \bfseries Some Conjectures on Fibonacci Numbers }\\[0.4cm]  %how do you make subtitles this is way too long
\HRule \\[1.5cm]
 
%----------------------------------------------------------------------------------------
%	AUTHOR SECTION
%----------------------------------------------------------------------------------------

\begin{minipage}{0.4\textwidth}
\begin{flushleft} \large
\emph{Authors:}\\
Daniel \textsc{Whatley}\\
Sarah \textsc{Sahibzada}\\
Taylor \textsc{Wilson}
\end{flushleft}
\end{minipage}
~
\begin{minipage}{0.4\textwidth}
\begin{flushright} \large
\emph{Supervisor:} \\
Dr. Sara \textsc{Pollock} 
\end{flushright}
\end{minipage}\\[4cm]

%----------------------------------------------------------------------------------------
%	DATE SECTION
%----------------------------------------------------------------------------------------

{\large \today}\\[3cm] 

%----------------------------------------------------------------------------------------
%	LOGO SECTION
%----------------------------------------------------------------------------------------

%\includegraphics[scale=.3]{tamulogo.png}\\[1cm] 
 
%----------------------------------------------------------------------------------------

\vfill 

\end{titlepage}

\tableofcontents
\newpage
\newpage

\section{Introduction}

The \textit{Fibonacci sequence}, $\{F_n\}$, is defined as follows: \[ F_1 = 1, \qquad F_2 = 1, \qquad F_n = F_{n - 1} + F_{n - 2} \quad \text{for} \quad n \geq 3. \] In this paper we analyze two conjectures: one about which Fibonacci numbers are perfect squares, and one about factorials (put reference and more info here).

\section{Theoretical Analysis}

It was a long-standing conjecture, until it was proved in 1964 [2], that the only Fibonacci numbers that are perfect squares are $F_1 = 1$, $F_2 = 1$, and $F_12 = 144$. M. Wunderlich[1] described an ingenious ``exclusion'' method to calculate which Fibonacci numbers, out of the first million, can possibly be perfect squares. The method does not prove that any Fibonacci numbers are perfect squares, rather it rules out those that are not.

Consider a prime $p$. We first calculate the period of the Fibonacci numbers $\pmod p$, denoted here as the \textit{Fibonacci period} of $p$. For example, the Fibonacci period of $p = 7$ is 16, as the Fibonacci sequence $\pmod 7$ repeats after 16 steps. Specifically, the first 16 Fibonacci numbers $\pmod{7}$ are: \[ 1, 1, 2, 3, 5, 1, 6, 0, 6, 6, 5, 4, 2, 6, 1, 0, \] after which it repeats $1, 1, 2, 3, 5, \dots$ again.

Take those residues $\pmod{p}$ that are quadratic non-residues $\pmod{p}$. If any Fibonacci number is congruent to a quadratic non-residue $\pmod{7}$, then it can never be a perfect square of any integer. Repeat, until the number of possible square Fibonacci numbers is down to a reasonable number to analyze. 

We also created a data structure to check Lucas number and factorial conjecture... (insert more detail)

\section{Computational Approach}

[1] found that the only Fibonacci numbers up to $F_{1000000}$ are the three described above using a computational approach. The researchers here took a similar approach, but used optimization techniques to decrease running time and to give more conclusive results. The programming language used was predominantly C++.

Functions to check which numbers are quadratic residues mod a certain prime, to calculate GCDs and LCMs, and to calculate the power of a number modulo another were implemented. The basic algorithm from [1] was also implemented as a preliminary version.

\section{Results}



\section{Conclusion}



\section{Individual Contributions}

\begin{enumerate}
\item Sarah:

\item Daniel: I wrote a significant portion of the code for 

\item Taylor:
\end{enumerate}

\section{References}


\end{document}
