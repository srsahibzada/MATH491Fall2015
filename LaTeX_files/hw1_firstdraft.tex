\documentclass{article}
\usepackage{amsmath,amssymb,amsthm,latexsym,paralist}

\DeclareRobustCommand{\stirling}{\genfrac\{\}{0pt}{}}

\theoremstyle{definition}
\newtheorem{problem}{Problem}
\newtheorem*{solution}{Solution}
\newtheorem*{resources}{Resources}

\newcommand{\name}[1]{\noindent\textbf{Name: Sarah Sahibzada}}
\newcommand{\honor}{\noindent On my honor, as an Aggie, I have neither
  given nor received any unauthorized aid on any portion of the
  academic work included in this assignment. Furthermore, I have
  disclosed all resources (people, books, web sites, etc.) that have
  been used to prepare this homework. \\[1ex]
 \textbf{Signature:} Sarah R. Sahibzada }

 
\newcommand{\checklist}{\noindent\textbf{Checklist:}
\begin{compactitem}[$\Box$] 
\item Did you add your name? 
\item Did you disclose all resources that you have used? \\
(This includes all people, books, websites, etc. that you have consulted)
\item Did you sign that you followed the Aggie honor code? 
\item Did you solve all problems? 
\item Did you submit (a) your latex source file and (b) the resulting pdf file
  of your homework?
\item Did you submit (c) a hardcopy of the pdf file in class? 
\end{compactitem}
}

\newcommand{\problemset}[1]{\begin{center}\textbf{Problem Set #1}\end{center}}
\newcommand{\duedate}[2]{\begin{quote}\textbf{Due dates:} Electronic submission of .tex
    and .pdf files of this homework is due on \textbf{#1} on e-campus
    (as a turnitin assignment), a signed paper copy
    of the pdf file is due on \textbf{#2} at the beginning of
    class. \end{quote} }

\newcommand{\N}{\mathbf{N}}
\newcommand{\R}{\mathbf{R}}
\newcommand{\Z}{\mathbf{Z}}


\begin{document}
\problemset{1}
\centerline{CSCE 411-502 (Dr. Klappenecker) }

\duedate{9/2/2015 before 11:00am}{9/2/2015}
\name{ (put your name here)}
\begin{resources} (All people, books, articles, web pages, etc. that
  have been consulted when producing your answers to this homework)
\end{resources}
\honor

\newpage

Get familiar with \LaTeX. Watch the (optional) video. All exercises
are from the lecture notes that were (or will) be handed out in
class. 
\begin{problem} (10 points)
Exercise 7.2. 
\end{problem}
\begin{solution}
\[H_{x} = 1 + 2 + . . . \frac{1}{x} = \sum\limits_{i=1}^x \frac{1}{x}\]
By definition, \[\Delta f(x) = f(x+1) - f(x) \]
Therefore, \[\Delta H_{x} = H_{x+1} - H_{x}\]
\[=\sum\limits_{i=1}^{x+1} \frac{1}{x} - \sum\limits_{i=1}^x \frac{1}{x}\]
\[=\frac{1}{x+1} + \sum\limits_{i=1}^x \frac{1}{x} - \sum\limits_{i=1}^x \frac{1}{x} \]
\[=\frac{1}{x+1}\]
It has been shown above that \[\Delta H_{x} = \frac{1}{x+1} \] . 
By the linearity of the difference operator, the statement \[\Delta (H_{x}\times x-x)\] is equivalent to \[\Delta (H_{x}\times x) - \Delta x\ = \Delta (H_{x}\times x) -1\]
Solving for the first term, we obtain \[\Delta H_{x}S_{x}- H_{x} \Delta{x}\]
\[= \frac{1}{x+1}\times (x+1) + H_{x}\]
Substituting this into the original equation:
\[H_{x} + 1 - 1 = H_{x}\]
Thus,\[\Delta((H_{x}\times x) - x) = H_{x}\]

\end{solution}

\begin{problem} (10 points)
Exercise 7.4.
\end{problem}
\begin{solution}

By definition of falling factorial, \[x^{\underline{k}} = x(x-1)...(x-(k+1))\].
Applying the identity \[\frac{x^{\underline{k+1}}}{x^{\underline{k}}} = x - k\] to the initial equation and re-arranging factors:\[\frac{x^{\underline{k+1}} + kx^{k}}{x^{\underline{k}}}  = \frac{x^{\underline{k+1}}}{x^{\underline{k}}} + \frac{kx^{\underline{k}}}{x^{\underline{k}}}\]
\[= (x - k) + k = x\]
Applying the same identity to the right-hand side of the equation, we obtain \[\frac{xx^{\underline{k}}}{x^{\underline{k}}} = x\]
Therefore, through application of this identity, both the left and right hand sides of this equation are equivalent. Thus, it can be said that \[x^{\underline{k+1}}+kx^{\underline{k}} = xx^{\underline{k}}\]
\end{solution}

\begin{problem} (10 points)
Exercise 7.8. 
\end{problem}
\begin{solution}
It is known by the definition of the difference operator that \[\Delta x^n = (x+1)^n - x^n\]. The binomial theorem may be used to expand this polynomial: \[ (x+1)^n - x = \bigg(\sum\limits_{k=0}^n {{n \choose {k}} (x+1)^{n-k}}\bigg) - x^n\]. It may be shown by induction on n that this is the expression for \[\Delta x^n\] for all positive integers n. Indeed, we can see that this holds for the case n = 1: \[\Delta x^1 = \Delta x = 1 = (x + 1)^1 - x\]. We may now state our inductive hypothesis:\[ (x+1)^n - x = \bigg(\sum\limits_{k=0}^n {{n \choose {k}} (x+1)^{n-k}}\bigg) - x^n\]. By induction, we seek to prove that for n + 1, we obtain: \[=\bigg(\sum\limits_{k=0}^{n+1} {{n+1 \choose {k}} (x+1)^{(n+1)-k}}\bigg) - x^{n+1}\]
This is shown below:
\[\Delta x^{n+1} = (x+1)^{n+1} - x^{n+1}\]
\[= (x+1)(x+1)^n - x^{n+1}\]
\[= (x+1)\bigg(\sum\limits_{k=0}^n {{n \choose {k}} (x+1)^{n-k}}\bigg) - x^{n+1}\]
\[=\]


\end{solution}

\begin{problem} (10 points)
Exercise 7.14.
\end{problem}
\begin{solution}
Applying the difference operator on both sides of the equation:
\[\Delta\big(\Delta^{-1}\big(\frac{1}{2^x}\big)\big) = \Delta\big(\frac{-1}{2^{x-1}}+C\big) \]
\[\frac{1}{2^{x}} = \Delta\big(\frac{-1}{2^{x-1}}+C\big)\]
\[\Delta\frac{-1}{2^{x}} + \Delta C = \Delta\frac{-1}{2^{x-1}}+0 = \frac{-1}{2^{x}} + \frac{1}{2^{x-1}}\]
\[= - \frac{1}{2(2^{x-1})} +\frac{1}{2^{x-1}} = \big(\frac{1}{2^{x-1}} \big)\big(-\frac{1}{2}+1 \big) \]
\[= \big(\frac{1}{2}\big)\big(\frac{1}{2^{x-1}}\big) = \frac{1}{2^{x}} \]
\end{solution}

\begin{problem} (10 points)
Exercise 7.16 (use finite difference calculus).
\end{problem}
\begin{solution}
\[\Delta^{-1} x^{2} = \frac{S(2,2)}{3}x^{\underline{3}} + \frac{S(2,1)}{2}x^{\underline{2}}+ \frac{S(2,0)}{1}x^{\underline{1}}\]
\[= \frac{1}{3}(x)(x-1)(x-2) + \frac{1}{2}(x)(x-1)\]
\[\sum\limits_{k=1}^n k^{2} = \frac{1}{3}(n+1)^{\underline{3}} + \frac{1}{2}(n+1)^{\underline{2}} = \frac{1}{3}(n+1)(n)(n-1_ + \frac{1}{2}(n+1)(n)\]
\[\frac{2}{6}(n^{2}-1)(n) + \frac{3}{6}(n^{2}+n) = \frac{2}{6}(n^{3}-n) + \frac{3}{6}(n^{2}+n)\]
\[= \frac{2}{6}n^{3} - \frac{2}{6}n + \frac{3}{6}n^{2} + \frac{3}{6}n = \frac{2n^{3} + 3n^{2} + n}{6}\]
\[=\frac{(n)(2n^{2}+3n+1)}{6} \]
\[=  \frac{(n)(2n+1)(n+1)}{6}\]
\end{solution}

\begin{problem} (10 points)
Exercise 7.17.
\end{problem}
\begin{solution}
Note that the sum \[\frac{1}{{1\times{2}}} + \frac{1}{2\times{3}} + . . .\frac{1}{n\times{(n+1)}}= \sum\limits_{k=0}^{n-1} x^{\underline{-2}}\] since this will have first term \[\frac{1}{1\times{2}}\] and nth term \[\frac{1}{n\times{(n+1)}}\].
We proceed as in the previous problem:
\[\Delta^{-1} x^{\underline{-2}} = \frac{1}{-2+1} x^{\underline{-2+1}} = -x^{\underline{-1}}  \]
Now, applying the fundamental theorem of summation, we get:
\[-n^{\underline{-1}} - 0^{\underline{-1}} = \frac{-1}{n+1} + 1\]
\[= 1 - \frac{1}{n+1}\]
\end{solution}

Watch the video on asymptotic notations. 
\begin{problem} (10 points)
Exercise 9.2.
\end{problem}
\begin{solution}
\[ln(exp(n^{a})) \texttt{\char`\~}  ln(exp(n+1)^{a})\] where \[ln(exp(n+1)^{a}) = (n+1)^{a}\] and \[ln(exp(n)) = n\]
\[\]
\end{solution}




I will allow that you explore some of the problems in class together with
your team, \textbf{but} the homework solution must be formulated by
yourself. Homeworks must be typeset in \LaTeX{}. 
\begin{solution}
\end{solution}








\goodbreak
\checklist
\end{document}
